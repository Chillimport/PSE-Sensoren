
\section{Klasse: Configuration}
Variablen:

\begin{itemize}
	\item name: String
	\item delimiter: Character
	\item headerlines: Integer
	\item staticTimezone: TimeZone
	\item variableTimezone: Integer
	\item dateTime: List<StringColumn>
 	\item streamData: List<StreamObservation>
\end{itemize}
\ \\
Methoden:
\begin{itemize}
	%TODO Methoden ergänzen
	\item Konstruktor: Configuration(name)
	\item Einfache Getter und Setter für:
	\begin{itemize}
		\item name
		\item delimiter
		\item headerlines
		\item dateTime
		\item streamData
	\end{itemize}

	\item Angepasste getter und setter:
	\begin{itemize}
		\item getStaticTimezone(): TimeZone
		\item getVariableTimezone(): Integer
		\item setStaticTimezone(TimeZone zone)
		\item setVariableTimezone(Integer column)
	\end{itemize}
\end{itemize}
\ \\
Beschreibung Allgemein: \\Diese Klasse stellt die Konfigurationen dar. Sie speichert
alle notwendigen Daten und bietet getter und setter für einen vereinfachten Zugriff auf Daten bereit.\\
\ \\
Beschreibung Variablen:
\begin{itemize}
	\item name:\\Hier wird der Name, eine eindeutige Kennung der Konfiguration, gespeichert.
	
	\item delimiter: \\Falls eine CSV-Datei importiert werden soll, muss ein Trennsymbol für die Spalten gewählt werden.
	
	\item headerlines: \\In einem CSV-Dokoment gibt es Kopfzeilen, welche für den Import irrelevant sind. Die Anzahl dieser Zeilen wird in dieser Variable gespeichert.
	
	\item staticTimezone: \\Falls die Zeitzone für alle Observations gleich ist, kann hier eine Zeitzone gewählt werden.
	
	\item variableTimezone: \\Hier wird die Spalte angegeben in der sich die Zeitzone befindet, falls die Zeitzone sich ändern kann
	
	\item dateTime:\\Hier wird eine Liste mit Datumsformaten mit einer zugehörigen Spalte gespeichert
	
	\item streamData\\ In dieser Liste werden die Datastreams zusammen mit den Spalten gespeichert, aus welchen die Daten extrahiert werden.
	
\end{itemize}
\ \\
Beschreibung Methoden:
\begin{itemize}
	\item getStaticTimezone(): TimeZone\\
	Gibt die gewählte Zeitzone oder NULL, falls die variableTimezone oder keine Timezone gewählt wurde
	
	\item getVariableTimezone(): Integer\\
	Gibt die Spalte zurück, in der sich die Zeitzone befinden soll.
	Gibt -1 zurück, falls keine Spalte oder eine StaticTimezone gewählt wurde
	
	\item setStaticTimezone(TimeZone zone) \\
	Setzt die StaticTimezone auf zone und die VariableTimezone auf -1
	
	\item setVariableTimezone(Integer column) \\
	Setzt die variableTimezone auf column und staticTimeZone auf NULL
\end{itemize}

\section{Klasse: StringColumn}
Variablen:
\begin{itemize}
	\item string: String
	\item column: Integer
\end{itemize} 

\section{Klasse: StreamObservation}
Variablen: 
\begin{itemize}
	\item dsID: String
	\item observation: List<Integer>
\end{itemize}

\section{ConfigManagement}
Variablen:
\begin{itemize}
	\item Keine Variablen auf die der Nutzer Einfluss haben könnte.\\
		Die Speicherpfade sind final und nicht einsehbar.
		
\end{itemize}
\ \\
Methoden:
\begin{itemize}
	\item loadConfig(String id): Config (JSON)
	\item saveConfig(Config (JSON))
	\item static: ConvertToJava(Config (JSON)): Configuration
	\item listAll(): List<String>
	\item getServer(): URL
\end{itemize}
\ \\
Beschreibung Allgemein:\\
Diese Klasse ist zuständig für die Verwaltung von Konfigurationen.

\ \\
Beschreibung Methoden:\\
\begin{itemize}
	\item loadConfig(String id): Config (JSON) \\
	Diese Methode bekommt den Namen einer Configuration übergeben, lädt die Configuration und gibt sie als JSON Objekt zurück.
	
	\item saveConfig(Config (JSON))\\
	Diese Methode bekommt als Eingabe eine Configuration als JSON Objekt
	und speichert dieses auf der Festplatte des Servers
	
	\item ConvertToJava(Config (JSON)): Configuration
	Um aus einer JSON Configuration eien Java Configuration zu erhalten übergibt man dieser Funktion ein JSON-Objekt und erhält als Ausgabe die gewünschte Java Configuration.
	
	\item listAll() \\
	Diese Funktion gibt eine Liste aller Konfigurationen, dargestellt als ihre Namen(String), zurück.
	
	\item getServer(): URL\\
	Diese Methode gibt die URL des Zielservers zurück 
\end{itemize}
