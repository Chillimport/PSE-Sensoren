    PSE Treffen
    
    System-Modell
    
    Importieren aus Zusatzfunktionen raus "extrahieren", Zentraler Hauptpunkt!
    Zusatzfunktionen detaillierter darstellen
    Controller lediglich als Schnittstelle von außen, hauptsächliche Logik woanders
    Ein/Aus als Extra-Paket und Funktionen als Hauptpaket
    Abfolgen untereinander und große Blöcke wie Logging/Fehler nebendran durchziehen
    Website holt sich Daten asynchron, User klickt -> Website holt daten   Also Website mehr Logik als "nur Darstellung"
    Website MVC im Kleinen
    Einzelne JS files etc nicht aufführen sondern Objektorientiert darstellen
    
    Klassenmodellierung
    
    Parser -> Abstraktes Modell kommt raus   (Damit z.b auch Json Parser nachgeschrieben werden kann, Programm verwendet intern Funtkionen auf dem abstrakten Modell weiter)
    Pipeline Modell nur nach vorne Date Parser -> Zone Parser etc z.b , keine zyklischen Abfolgen oder Rücksprünge
    Cut nur einmal implementieren (wurde schon bei parseCSV umgewandelt)
    (Interne Tabellendarstellung? Was muss sie können?Z.B Zugriff auf bestimmte Zeilen/Spalten? Methoden definieren, Zugriff durchspielen, überlegen an welcher Stelle Fehler auftreten können)
    Getter/Setter nicht explizit aufführen, public werte haben eh alle g/s
    Komplettübersicht über Architektur/Fluss wichtiger als einzelne Methoden etc
    
    FROST-Client Nutzung : https://github.com/hylkevds/SensorThingsStuff/tree/master/src/main/java/de/fraunhofer/iosb/ilt/tests
    (SensorThingsManager / SensorThingsImporter komplexeres Beispiel)
    
    Service Klasse legt Dinge direkt auf dem Server an (Authentifizierung beim Service öffnen weglassen), Locations/Things erstellen und mit Service anlegen.
    
    
    