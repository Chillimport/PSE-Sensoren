\subsection{ObservedPropertyController}

\paragraph{Beschreibung}
Diese Controller-Klasse empfängt Anfragen vom Frontend, die das Erstellen oder Abrufen von ObservedProperties betreffen. Zur Bearbeitung dieser Anfragen greift die Klasse auf den Frost-Client zu.


\paragraph{Attribute}

\paragraph{Methoden}
\begin{itemize}
\item[+] \textit{ create(ObservedProperty o): ResponseEntity<?> }
erstellt mithilfe des Frost-Client eine neue ObservedProperty und lädt diese auf den Frostserver
\item[+] \textit{get(int id): ResponseEntity<ObservedProperty>}
sucht auf dem Server die ObservedProperty mit der angegebenen id und gibt diese zurück (NULL im Fehlerfall)
\item[+] \textit{getAll(): ResponseEntity<List<ObservedProperty>> }
gibt alle auf dem FROST-Server gespeicherten ObservedProperties zurück
\end{itemize}