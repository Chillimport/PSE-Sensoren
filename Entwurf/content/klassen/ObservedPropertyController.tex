\subsection{ObservedPropertyController}

\paragraph{Beschreibung}
Diese Controller-Klasse empfängt Anfragen vom Frontend, die das Erstellen oder Abrufen von ObservedProperties betreffen. Zur Bearbeitung dieser Anfragen greift die Klasse auf den Frost-Client zu.


\paragraph{Attribute}

\paragraph{Methoden}
\begin{itemize}
\item[+] \textit{ createObservedProperty(String name, String description, String definition) : void }
erstellt mithilfe des Frost-Client eine neue ObservedProperty und lädt diese auf den Frostserver.
\item[+] \textit{getObservedProperty(int id) : ObservedProperty}
sucht auf dem Server die ObservedProperty mit der angegebenen id und gibt diese zurück (NULL im Fehlerfall)
\item[+] \textit{getObservedProperties(String begin) : List<ObservedProperty> }
sucht auf dem Frost-Server die ObservedProperties (maximal die ersten 7), deren Namen mit der angegeben Zeichenkette beginnen.
\end{itemize}