\subsection{SensorController}

\paragraph{Beschreibung}
Diese Controller-Klasse empfängt Anfragen vom Frontend, die das Erstellen oder Abrufen von Sensoren betreffen. Zur Bearbeitung dieser Anfragen greift die Klasse auf den Frost-Client zu.


\paragraph{Attribute}

\paragraph{Methoden}
\begin{itemize}
\item[+] \textit{ createSensor(String name,String description,EncodingType encodingType,String metadata) : void }
erstellt mithilfe des Frost-Client einen neuen Sensor und lädt diesen auf den Frostserver.
\item[+] \textit{getSensor(int id) : Sensor}
sucht auf dem Server den Sensor mit der angegebenen id und gibt diesen zurück (NULL im Fehlerfall)
\item[+] \textit{getSensors(String begin) : List<Sensor> }
sucht auf dem Frost-Server die Sensoren (maximal die ersten 7), deren Namen mit der angegeben Zeichenkette beginnen.
\end{itemize}