\subsection{ConfigurationController}

\paragraph{Beschreibung}
Diese Controller-Klasse empfängt Anfragen vom Frontend, die das Erstellen oder Abrufen von Konfigurationen betreffen. Zur Bearbeitung dieser Anfragen greift die Klasse auf den ConfigurationManager zu.


\paragraph{Attribute}

\paragraph{Methoden}
\begin{itemize}
\item[+] \textit{ createAndSaveConfiguration(???) : void }
erstellt mithilfe des ConfigurationManager eine neue Configuration und speichert sie ab.
\item[+] \textit{getConfiguration(int id) : JSON}
sucht auf dem Server das Thing mit der angegebenen id und gibt es zurück (NULL im Fehlerfall)
\item[+] \textit{getConfigurations(String begin) : JSON }
sucht aus dem Speicher die Configurations (maximal die ersten 7), deren Namen mit der angegeben Zeichenkette beginnen.
\end{itemize}