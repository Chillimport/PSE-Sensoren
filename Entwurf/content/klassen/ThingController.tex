\subsection{ThingController}

\paragraph{Beschreibung}
Diese Controller-Klasse empfängt Anfragen vom Frontend, die das Erstellen oder Abrufen von Things betreffen. Zur Bearbeitung dieser Anfragen greift die Klasse auf den Frost-Client zu.


\paragraph{Attribute}

\paragraph{Methoden}
\begin{itemize}
\item[+] \textit{ createThing(String name, String description, Map<String, Object> properties, Location location) : void }
erstellt mithilfe des Frost-Client ein neues Thing und lädt es auf den Frostserver.
\item[+] \textit{getThing(int id) : Thing}
sucht auf dem Server das Thing mit der angegebenen id und gibt es zurück (NULL im Fehlerfall)
\item[+] \textit{getThings(String begin) : List<Thing> }
sucht auf dem Frost-Server die Things (maximal die ersten 7), deren Namen mit der angegeben Zeichenkette beginnen.
\end{itemize}