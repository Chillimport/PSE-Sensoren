
\subsection{Configuration}
Diese Klasse stellt die Konfigurationen dar. 
Sie speichert alle notwendigen Daten und stellt getter für einen vereinfachten Zugriff auf Daten bereit.

\paragraph{Attribute}

\begin{itemize}
	\item \textit{private Integer id} Hier wird die ID der Konfiguration gespeichert.
	
	\item \textit{private Character delimiter} Falls eine CSV-Datei importiert werden soll, muss ein Trennsymbol für die Spalten gewählt werden.
	
	\item \textit{private Integer headerlines} In einem CSV-Dokoment gibt es Kopfzeilen, welche für den Import irrelevant sind. Die Anzahl dieser Zeilen wird hier gespeichert.
	
	\item \textit{private Integer timezone} Hier wird die Zeitzone gespeichert. Falls die Zahl positiv ist, handelt es sich um eine Spaltenangabe. Falls die Zahl negativ ist, handelt sich um die Integer-Darstellung einer konstanten Zeitzone.
	
	\item \textit{private List<StringColumn> dateTime} Hier wird eine Liste aus Datumsformaten mit einer zugehörigen Spalte gespeichert 
	
	\item \textit{private List<StreamObseration> streamData} In dieser Liste werden die Datastreams zusammen mit den Spalten gespeichert, aus welchen die Daten extrahiert werden.
\end{itemize}

\paragraph{Methoden}
\begin{itemize}
	\item \textit{public Configuration(Integer id, Character delimiter, Integer headerlines, Integer timezone, List<StringColumn> dateTime, List<StreamObservation> streamData)}
	
	\item \textit{public Integer getId()} Diese Methode gibt die id der Konfiguration zurück.
	
	\item \textit{public Character getDelimiter()} Diese Methode gibt den delimiter zurück.
	
	\item \textit{public Integer getHeaderlines()} Diese Methode gibt die Anzahl der Kopfzeilen zurück.
	
	\item \textit{public Integer getTimezone()} Diese Methode gibt timezone zurück.	
	
	\item \textit{public List<StringColumn> getDateTime()} Diese Methode gibt die DateTime-Liste zurück.
	
	\item \textit{public List<StreamObseration> getStreamData()} Diese Methode gibt die streamData zurück.

	\item \textit{public static Configuration ConvertToJava(JSONObject)]} Diese Methode generiert aus einem JSONObject eine Configuration.
	
	\item \textit{public static JSONObject ConvertToJSON(Configuration)]} Diese Methode generiert aus einer Konfiguration ein JSONObject.
\end{itemize}


\subsection{StringColumn}
Diese Klasse wird dazu verwendet einen String zusammen mit einer Spaltennummer zu speichern.
\paragraph{Attribute}
\begin{itemize}
	\item \textit{private String string} Ein String
	\item \textit{private Integer column} Eine Spaltennummer
\end{itemize} 

\paragraph{Methoden}
\begin{itemize}
	\item \textit{public void setString(String string)}
	\item \textit{public String getString()}
	
	\item \textit{public void setColumn(Integer column)}
	\item \textit{public Integer getColumn()}
\end{itemize}

\subsection{StreamObservation}
\paragraph{Attribute} 
\begin{itemize}
	\item \textit{private Integer dsID}
	\item \textit{private List<Integer> observation}
\end{itemize}

\paragraph{Methoden}
\begin{itemize}
	\item \textit{public void setDsID(Integer dsID)}
	\item \textit{public void setObservations(List<Integer>)}
	\item \textit{public Integer getDsID()}
	\item \textit{public List<Integer> getObservtions()}
\end{itemize}
