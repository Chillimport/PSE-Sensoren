\subsection{StringColumn}
Diese Klasse stellt ein Tupel aus einem String und einer Spaltennummer dar.
Diese Klasse wird dazu verwendet eine DateTime anzugeben. \\Diese hat einen Regex-String und eine Spaltennummer, in welcher der zu importierende Wert zu finden ist.
\paragraph{Attribute}
\begin{enumerate}[-]
	\item \textit{string : String} Ein String
	\item \textit{column : Integer} Eine Spaltennummer
\end{enumerate} 

\subsection{StreamObservation}
Diese Klasse speichert einen Datastream zusammen mit einer Liste aus Spaltenangaben, welche anzeigen, wo die Observations auszulesen sind.
\paragraph{Attribute} 
\begin{enumerate}[-]
	\item \textit{dsID : Integer} ID eines (Multi-) Datastreams. 
	\item \textit{observation : List<Integer>} Spaltenangaben.
\end{enumerate}

\subsection{ZoneIdColumn}
Diese Klasse stellt ein Tupel aus einer ZoneId und einer Spaltennummer dar.\\
Sie wird verwendet um die variableTimeZone mit der staticTimeZone zu verbinden, so wird diese Klasse zum Beispiel als Rückgabeparameter bei getTimezone verwendet.

\paragraph{Attribute}
\begin{enumerate}[-]
	\item \textit{zoneId : ZoneId} Die ID einer Zeitzone.
	\item \textit{column : Integer} Spaltennummer.
\end{enumerate}

\subsection{Configuration}
Diese Klasse stellt die Konfigurationen dar. 
Sie speichert alle notwendigen Daten und stellt "{getter}"{-Methoden} für einen vereinfachten Zugriff auf Daten bereit.

\paragraph{Attribute}

\begin{enumerate}[-]
	\item \textit{id : Integer} Hier wird die ID der Konfiguration gespeichert. Die ID wird automatisch generiert.
	
	\item \textit{name : String} Hier wird der Name der Konfiguration gespeichert.
	
	\item \textit{delimiter : Character} Falls eine CSV-Datei importiert werden soll, muss ein Trennsymbol für die Spalten gewählt werden.
	
	\item \textit{numberOfHeaderlines : Integer} In einem CSV-Dokument gibt es Kopfzeilen, welche für den Import irrelevant sind. Die Anzahl dieser Zeilen wird hier gespeichert.
	
	\item \textit{timezone : ZoneIdColumn} Hier wird die Zeitzone gespeichert.
	
	\item \textit{dateTime : List<StringColumn>} Hier wird eine Liste aus Datumsformaten mit einer zugehörigen Spalte gespeichert 
	
	\item \textit{streamData : List<StreamObseration>} In dieser Liste werden die Datastreams zusammen mit den Spalten gespeichert, aus welchen die Daten extrahiert werden.
	
	\item \textit{mapOfMagicNumbers : MagicNumberMap} Hier wird das Mapping für Sonderzeichen gespeichert. Sonderzeichen sollen nicht auf dem FROST-Server gespeichert werden und haben eine spezielle Bedeutung.
\end{enumerate}

\paragraph{Konstruktoren}
\begin{enumerate}[+]
	\item \textit{Configuration(Integer id, String name, Character delimiter, Integer numberOfHeaderlines, ZoneIdColumn timezone, List<StringColumn> dateTime, List<StreamObservation> streamData, MagicNumberMap mapOfMagicNumbers)} Dieser Konstruktor erstellt aus den gegebenen Parametern eine vollständig ausgefüllte Configuration.
	\begin{enumerate}[$\bullet$]
		\item \textit{Integer id:} ID der Konfiguration.
		\item \textit{String name:} Name der Konfiguration.
		\item \textit{Character delimiter:} Spaltentrennsymbol.
		\item \textit{Integer numberOfHeaderlines:} Anzahl der Kopfzeilen.
		\item \textit{ZoneIdColumn timezone:} Zeitzone oder Splatennummer der Zone.
		\item \textit{List<StringColumn> dateTime:} Liste für Zeitangaben.
		\item \textit{List<StreamObservation> streamData:} Liste aus Streams mit dazugehörigen Spaltennummern für Observations.
		\item \textit{MagicNumberMap mapOfMagicNumbers:} Mapping für Sonderzeichen.
	\end{enumerate}
\end{enumerate}

\paragraph{Methoden}
\begin{enumerate}[+]
	\item \textit{\underline{convertToJava(String jsonString) : Configuration}} Diese Methode generiert aus einem JSON-String eine Configuration.
	\begin{enumerate}[$\bullet$]
		\item \textit{String jsonString:} Der String, in dem das JSON-Objekt kodiert ist.
	\end{enumerate}
	\begin{enumerate}[$\circ$]
		\item \textit{String Configuration:} Die generierte Konfiguration.
	\end{enumerate}
	
	\item \textit{\underline{serialize(Configuration config) : String}} Diese Methode generiert aus einer Konfiguration ein JSON-String.
	\begin{enumerate}[$\bullet$]
		\item \textit{Configuration config:} Die Konfiguration, welche in einen String überführt werden soll.
	\end{enumerate}
	\begin{enumerate}[$\circ$]
		\item \textit{String:} Das generierte JSON-Objekt als String.
	\end{enumerate}
\end{enumerate}
