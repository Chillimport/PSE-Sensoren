\subsection{Table}

Repräsentiert eine 2-dimensionale Tabelle bestehend aus Zeilen und Spalten mit dem Fokus auf zeilenweise Verarbeitung. Der Inhalt der Tabellenelemente ist nicht beschränkt.


\paragraph{Attribute}

\begin{itemize}
	\item[-] \textit{LinkedList<ArrayList<Object>> table} Speichert die Daten der Tabelle, jedes LinkedList-Element enthält eine ArrayListe, welche die Elemente der Tabelle hält
	\item[-] \textit{String id}	Die Identifikationsnummer der Tabelle, kann der Name sein
\end{itemize}

\paragraph{Konstruktoren}

\begin{itemize}
	\item[+] \textit{Table(Table t, String id): void} Erstellt eine Table-Instanz aus einer 2-dimensionalen Liste \\
	Dabei werden alle inneren Listen auf die selbe Länge verlängert und mit NULL aufgefüllt, damit die Tabelle immer gleich viele Spalten besitzt
	\begin{enumerate}
		\item \textit{Table t}: Die Table-Instanz aus der ein neuer Table erstellt wird, kann NULL sein
		\item \textit{String id}: Die ID der neuen Instanz, kann NULL sein für keine ID
	\end{enumerate}

	\item[+] \textit{Table(String id): void} Erstellt einen neuen Table mit leerer Tabelle
	\begin{enumerate}
		\item \textit{String id}: Die ID der neuen Instanz, kann NULL sein für keine ID
	\end{enumerate}
\end{itemize}	
	
		

			
\paragraph{Methoden}
\begin{itemize}
	\item[+] \textit{get(int row, int column): Object} \\
	Gibt ein einzelnes Element der Tabelle an der angegebenen Stelle zurück
	\begin{enumerate}
		\item \textit{int row}: Die Zeile an der das Element liegt, darf nicht NULL sein
		\item \textit{int column}: Die Spalte an der das Element liegt, darf nicht NULL sein
		\begin{itemize}
			\item[] \textit{Object}: Das Element an der Stelle
		\end{itemize}
	\end{enumerate}

	\item[+] \textit{get(int row): List<Object>} \\
	Gibt eine ganze Zeile zurück
	\begin{enumerate}
		\item \textit{int row}: Die Position der Zeile, darf nicht NULL sein
		\begin{itemize}
			\item[] \textit{List<Object>}: die Zeile als Liste
		\end{itemize}
	\end{enumerate}

	\item[+] \textit{get(int column): List<Object>} \\
	Gibt eine ganze Spalte zurück
	\begin{enumerate}
	\item \textit{int column}: Die Position der Spalte, darf nicht NULL sein
		\begin{itemize}
			\item[] \textit{List<Object>}: die Spalte als Liste
		\end{itemize}
	\end{enumerate}	
		
	\item[+] \textit{getFirstRow(): List<Object>} \\
	Gibt die erste Zeile zurück
	\begin{enumerate}
		\item \textit{Keine Parameter}
		\begin{itemize}
			\item[] List<Object>: Die erste Zeile
		\end{itemize}
	\end{enumerate}	

	\item[+] \textit{getLastRow(): List<Object>} \\
	Gibt die letzte Zeile zurück
		\begin{enumerate}
			\item \textit{Keine Parameter}
			\begin{itemize}
				\item[] List<Object>: Die letzte Zeile
			\end{itemize}
		\end{enumerate}
	
	\item[+] \textit{set(Object o, int row, int column): void} \\
	Ersetzt ein Element der Tabelle mit einem neuen Element
	\begin{enumerate}
		\item \textit{Object o}: Das neue Element, darf NULL sein
		\item \textit{int row}: Die Zeile des Elements
		\item \textit{int column}: Die Spalte des Elements
	\end{enumerate}

	\item[+] \textit{set(List<Object> row, int pos): void} \\
	Ersetzt eine Zeile mit einer neuen Zeile
	\begin{enumerate}
		\item \textit{List<Object> row}: Die neue Zeile als Liste, darf NULL sein
		\item \textit{int pos}: Die Position der zu ersetzenden Zeile
	\end{enumerate}

	\item[+] \textit{set(List<Object> column, int pos): void} \\
	Ersetzt eine Spalte mit einer neuen Spalte
	\begin{enumerate}
		\item \textit{List<Object> column}: Die neue Spalte als Liste, darf NULL sein
		\item \textit{int pos}: Die Position der zu ersetzen Spalte
	\end{enumerate}

	\item[+] \textit{remove(int row, int column): void} \\
	Ersetzt ein Element mit NULL
	\begin{enumerate}
		\item \textit{int row}: Die Zeile des Elements
		\item \textit{int column}: Die Spalte des Elements
	\end{enumerate}

	\item[+] \textit{remove(int row): void} \\
	Ersetzt eine Zeile mit NULL
	\begin{enumerate}
		\item \textit{int row}: Die Position der Zeile
	\end{enumerate}

	\item[+] \textit{remove(int column): void} \\
	Ersetzt eine Spalte mit NULL
	\begin{enumerate}
		\item \textit{int column}: Die Position der Spalte
	\end{enumerate}

	\item[+] \textit{insert(List<Object> row, int position): void} \\
	Fügt eine neue Zeile hinter einer bestehenden Zeile ein
	\begin{enumerate}
		\item \textit{List<Object> row}: Die einzufügende Zeile als Liste
		\item \textit{int position}: Die Position hinter der die Zeile eingefügt werden soll
	\end{enumerate}

	\item[+] \textit{insert(List<Object> column, int position): void} \\
	Fügt eine neue Spalte hinter einer bestehenden Spalte ein
	\begin{enumerate}
		\item \textit{List<Object> column}: Die einzufügende Spalte als Liste
		\item \textit{int position}: Die Position hinter der die Spalte eingefügt werden soll
	\end{enumerate}

	\item[+] \textit{getRowCount(): int} \\
	Gibt die Anzahl Zeilen zurück
	\begin{enumerate}
		\item \textit{Keine Parameter}
		\begin{itemize}
			\item[] \textit{int}: Die Anzahl Zeilen
		\end{itemize}
	\end{enumerate}

	\item[+] \textit{getColumnCount(): int} \\
	Gibt die Anzahl Spalten zurück
	\begin{enumerate}
		\item \textit{Keine Parameter}
		\begin{itemize}
			\item[] \textit{int}: Die Anzahl Spalten
		\end{itemize}
	\end{enumerate}
	
	\item[+] \textit{getID(): String} \\
	Gibt die ID der Tabelle zurück
	\begin{enumerate}
		\item \textit{Keine Parameter}
		\begin{itemize}
			\item[] \textit{String}: Die ID, NULL falls keine ID gesetzt wurde
		\end{itemize}
	\end{enumerate}
	
	\item[+] \textit{clone(): Table} \\
	Erstellt eine ganze Kopie der Tabelle
	\begin{enumerate}
		\item \textit{Keine Parameter}
		\begin{itemize}
			\item[] \textit{Table}: die Kopie der Tabelle als neue Table-Instanz
		\end{itemize}
	\end{enumerate}

	\item[+] \textit{clear(): void} \\
	Löscht die Tabelle komplett und ersetzt sie gegen NULL
	
	\item[+] \textit{isEmpty(): boolean} \\
	Gibt zurück ob die Tabelle leer ist
	\begin{enumerate}
		\item \textit{Keine Parameter}
		\begin{itemize}
			\item[] \textit{boolean}: true wenn die Tabelle leer oder NULL ist
		\end{itemize}
	\end{enumerate}

	\item[+] \textit{compare(Table o): boolean} \\
	Vergleicht die Table-Instanz mit einer anderen Instanz
	\begin{enumerate}
		\item \textit{Table o}: Die andere zu vergleichede Table-Instanz
		\begin{itemize}
			\item[] \textit{boolean}: true, wenn alle Einträge gleich sind, sonst false
		\end{itemize}
	\end{enumerate}
\end{itemize}