\subsection{Table}
Repräsentiert eine 2-dimensionale Tabelle bestehend aus Zeilen und Spalten mit dem Fokus auf zeilenweise Verarbeitung. Der Inhalt der Tabellenelemente ist nicht beschränkt.


\paragraph{Attribute}

\begin{itemize}
	\item[-] \textit{List{<List<Object>}> table} \\
	Speichert die Daten der Tabelle, jedes Listenelement enthält eine weitere Liste, welche die Elemente der Tabelle hält
\end{itemize}

\paragraph{Konstruktoren}

\begin{enumerate}[+]
	\item \textit{Table(Table t): void} \\
	Erstellt eine Table-Instanz aus einer 2-dimensionalen Liste \\
	Dabei werden alle inneren Listen auf die selbe Länge verlängert und mit NULL aufgefüllt, damit die Tabelle immer gleich viele Spalten besitzt	
	\begin{enumerate}[$\bullet$]
		\item \textit{Table t}: Die Table-Instanz aus der ein neuer Table erstellt wird, kann NULL sein
	\end{enumerate}
	\vspace{-0.2cm}
	\item \textit{Table(): void} \\
	Erstellt einen neuen Table mit leerer Tabelle
\end{enumerate}
			
\paragraph{Methoden}

\begin{enumerate}[+]
	\item \textit{getElement(int row, int column): Object} \\
	Gibt ein einzelnes Element der Tabelle an der angegebenen Stelle zurück
	\begin{enumerate}[$\bullet$]
\item \textit{int row}: Die Zeile an der das Element liegt, darf nicht NULL sein
\item \textit{int column}: Die Spalte an der das Element liegt, darf nicht NULL sein
	\end{enumerate}
	\vspace{-0.2cm}
	\begin{enumerate}[$\circ$]
		\item \textit{Object}: Das Element an der Stelle
	\end{enumerate}
	
	\item \textit{getRow(int row): List<Object>} \\
	Gibt eine ganze Zeile zurück
	\begin{enumerate}[$\bullet$]
		\item \textit{int row}: Die Position der Zeile, darf nicht NULL sein
	\end{enumerate}
	\vspace{-0.2cm}
	\begin{enumerate}[$\circ$]
		\item \textit{List<Object>}: die Zeile als Liste
	\end{enumerate}
	
	\item \textit{getColumn(int column): List<Object>} \\
	Gibt eine ganze Spalte zurück
	
	\begin{enumerate}[$\bullet$]
		\item \textit{int column}: Die Position der Spalte, darf nicht NULL sein
	\end{enumerate}
	\vspace{-0.2cm}
	\begin{enumerate}[$\circ$]
		\item \textit{List<Object>}: die Spalte als Liste
	\end{enumerate}
	
	\item \textit{getFirstRow(): List<Object>} \\
	Gibt die erste Zeile zurück
	\vspace{-0.2cm}
	\begin{enumerate}[$\circ$]
		\item \textit{List<Object>}: Die erste Zeile
	\end{enumerate}

	\item \textit{getLastRow(): List<Object>} \\
	Gibt die letzte Zeile zurück
	\vspace{-0.2cm}
	\begin{enumerate}[$\circ$]
		\item \textit{List<Object>}: Die letzte Zeile
	\end{enumerate}

	\item \textit{setElement(Object o, int row, int column): void} \\
	Ersetzt ein Element der Tabelle mit einem neuen Element
	\begin{enumerate}[$\bullet$]
		\item \textit{Object o}: Das neue Element, darf NULL sein
		\item \textit{int row}: Die Zeile des Elements
		\item \textit{int column}: Die Spalte des Elements
	\end{enumerate}
	\vspace{-0.2cm}
	
	\item \textit{setRow(List<Object> row, int pos): void} \\
	Ersetzt eine Zeile mit einer neuen Zeile
	\begin{enumerate}[$\bullet$]
		\item \textit{List<Object> row}: Die neue Zeile als Liste, darf NULL sein
		\item \textit{int pos}: Die Position der zu ersetzenden Zeile
	\end{enumerate}
	\vspace{-0.2cm}

	\item \textit{setColumn(List<Object> column, int pos): void} \\
	Ersetzt eine Spalte mit einer neuen Spalte
	\begin{enumerate}[$\bullet$]
		\item \textit{List<Object> column}: Die neue Spalte als Liste, darf NULL sein
		\item \textit{int pos}: Die Position der zu ersetzen Spalte
	\end{enumerate}
	\vspace{-0.2cm}

	\item erste public Methode\\
	beschreibung der Methode
	\begin{enumerate}[$\bullet$]
		\item \textit{param 1:} beschreibung
		\item \textit{param 2:} beschreibung
	\end{enumerate}
	\vspace{-0.2cm}
	\begin{enumerate}[$\circ$]
		\item \textit{Rückgabeparameter:} beschreibung
	\end{enumerate}

	\item \textit{remove(int row, int column): void} \\
	Ersetzt ein Element mit NULL
	\begin{enumerate}[$\bullet$]
		\item \textit{int row}: Die Zeile des Elements
		\item \textit{int column}: Die Spalte des Elements
	\end{enumerate}
	\vspace{-0.2cm}
	
	\item \textit{insertRow(List<Object> row, int position): void} \\
	Fügt eine neue Zeile hinter einer bestehenden Zeile ein
	\begin{enumerate}[$\bullet$]
		\item \textit{List<Object> row}: Die einzufügende Zeile als Liste
		\item \textit{int position}: Die Position hinter der die Zeile eingefügt werden soll
	\end{enumerate}
	\vspace{-0.2cm}

	\item \textit{insertColumn(List<Object> column, int position): void} \\
	Fügt eine neue Spalte hinter einer bestehenden Spalte ein
	\begin{enumerate}[$\bullet$]
		\item \textit{List<Object> column}: Die einzufügende Spalte als Liste
		\item \textit{int position}: Die Position hinter der die Spalte eingefügt werden soll
	\end{enumerate}
	\vspace{-0.2cm}

	\item \textit{getRowCount(): int} \\
	Gibt die Anzahl Zeilen zurück
	\vspace{-0.2cm}
	\begin{enumerate}[$\circ$]
		\item \textit{int}: Die Anzahl Zeilen
	\end{enumerate}

	\item \textit{getColumnCount(): int} \\
	Gibt die Anzahl Spalten zurück
	\vspace{-0.2cm}
	\begin{enumerate}[$\circ$]
		\item \textit{int}: Die Anzahl Spalten
	\end{enumerate}

	\item \textit{getColumnCount(): int} \\
	Gibt die Anzahl Spalten zurück
	\vspace{-0.2cm}
	\begin{enumerate}[$\circ$]
		\item \textit{int}: Die Anzahl Spalten
	\end{enumerate}

	\item \textit{clone(): Table} \\
	Erstellt eine ganze Kopie der Tabelle
	\vspace{-0.2cm}
	\begin{enumerate}[$\circ$]
		\item \textit{Table}: die Kopie der Tabelle als neue Table-Instanz
	\end{enumerate}

	\item \textit{clear(): void} \\
	Löscht die Tabelle komplett und ersetzt sie gegen NULL

	\item \textit{isEmpty(): boolean} \\
	Gibt zurück ob die Tabelle leer ist
	\vspace{-0.2cm}
	\begin{enumerate}[$\circ$]
		\item \textit{boolean}: true wenn die Tabelle leer oder NULL ist
	\end{enumerate}

	\item \textit{compare(Table o): boolean} \\
	Vergleicht die Table-Instanz mit einer anderen Instanz
	\begin{enumerate}[$\bullet$]
		\item \textit{Table o}: Die andere zu vergleichede Table-Instanz
	\end{enumerate}
	\vspace{-0.2cm}
	\begin{enumerate}[$\circ$]
		\item \textit{boolean}: true, wenn alle Einträge gleich sind, sonst false
	\end{enumerate}
\end{enumerate}