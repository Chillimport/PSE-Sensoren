
\subsection{ConfigManagement}
Variablen:
\begin{itemize}
	\item Keine Variablen auf die der Nutzer Einfluss haben könnte.\\
		Die Speicherpfade sind final und nicht einsehbar.
		
\end{itemize}
\ \\
Methoden:
\begin{itemize}
	\item loadConfig(String id): Config (JSON)
	\item saveConfig(Config (JSON))
	\item static: ConvertToJava(Config (JSON)): Configuration
	\item listAll(): List<String>
	\item getServer(): URL
\end{itemize}
\ \\
Beschreibung Allgemein:\\
Diese Klasse ist zuständig für die Verwaltung von Konfigurationen.

\ \\
Beschreibung Methoden:\\
\begin{itemize}
	\item loadConfig(String id): Config (JSON) \\
	Diese Methode bekommt den Namen einer Configuration übergeben, lädt die Configuration und gibt sie als JSON Objekt zurück.
	
	\item saveConfig(Config (JSON))\\
	Diese Methode bekommt als Eingabe eine Configuration als JSON Objekt
	und speichert dieses auf der Festplatte des Servers
	
	\item ConvertToJava(Config (JSON)): Configuration
	Um aus einer JSON Configuration eien Java Configuration zu erhalten übergibt man dieser Funktion ein JSON-Objekt und erhält als Ausgabe die gewünschte Java Configuration.
	
	\item listAll() \\
	Diese Funktion gibt eine Liste aller Konfigurationen, dargestellt als ihre Namen(String), zurück.
	
	\item getServer(): URL\\
	Diese Methode gibt die URL des Zielservers zurück 
\end{itemize}
