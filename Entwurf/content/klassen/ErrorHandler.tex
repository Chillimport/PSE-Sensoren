\subsection{ErrorHandler}

\paragraph{Beschreibung}
Diese Klasse nimmt Fehlermeldungen entgegen und speichert sie ab. Es kann nur maximal eine Kopie der Klasse existieren. (Singleton)

\paragraph{Attribute}

\begin{enumerate}[$\bullet$]
	\item \underline{\textit{final ErrorHandler errorHandler}} \\
	Referenz auf die einzige ErrorHandler-Instanz
	\item \underline{\textit{final String path}} \\
	Speicherpfad der Datei in die die Fehlerhaften Ursprungs-Zeilen kopiert werden
	\item \textit{List <skippedRow> skippedRows} \\
	Die Liste von Zeilen und zugehörigen Exceptions
\end{enumerate}

\paragraph{Konstruktoren}
\begin{enumerate}[-]
	\item \textit{ErrorHandler()} \\
	Privater Konstruktor für das Singleton-Element
\end{enumerate}

\paragraph{Methoden}

\begin{enumerate}[+]
	\item \underline{\textit{getInstance(): ErrorHandler}} \\
	Gibt die einzige ErrorHandler-Instanz zurück
	\begin{enumerate}[$\circ$]
 		\item \textit{ErrorHandler} Die ErrorHandler-Instanz
	\end{enumerate}

	\item \textit{addRows(Integer row, Exception e): void} \\
	Fügt ein neues skippedRow-Objekt zur Liste hinzu mit der übersprungenen Zeile und der Exception als Information 
	\begin{enumerate}[$\bullet$]
		\item \textit{Integer row} Die Zeile
		\item \textit{Exception e} Der aufgetretene Fehler
	\end{enumerate}

	\item \textit{returnRows(): void} Kopiert alle übersprungenen Zeilen aus der Originaldatei und gibt sie in einer neuen Datei zurück.
\end{enumerate}

