\subsection{ErrorHandler}

\paragraph{Beschreibung}
Diese Klasse nimmt Fehlermeldungen entgegen und behandelt sie (wenn möglich).


\paragraph{Methoden}

\begin{itemize}
\item + notify(String message) : void
Informiert den User über ein Ereignis, entweder durch eine Popup Message oder über das Log-Fenster (Je nach Dringlichkeit)

\item + identify(String message) : boolean
Stellt fest um welchen Fehler es sich handelt und leitet entsprechende Maßnahmen ein. (z.b Zeile überspringen)
Gibt zurück ob der Fehler behandelt werden konnte oder nicht.

\item + identify(Exception e) : boolean
Identifiziert den Exception Typ und behandelt die Exception.(I.d.r Zeile überspringen)
Gibt zurück ob die Exception behandelt werden konnte.

\item - skipRow() : void
Sort dafür, dass die aktuelle Zeile übersprungen wird.
\end{itemize}