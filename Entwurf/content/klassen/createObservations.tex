
\subsection{CreateObservations}

\paragraph{Beschreibung}
Hier werden die Observations erstellt, die anschließend auf den Frost-Server hochgeladen werden können.
Dazu werden die Daten aus der internen Tabellendarstellung gelesen und daraus zeilenweise Observations erstellt, mit der Duplikatklasse auf Duplikate überprüft und anschließend hochgeladen

\paragraph{Attribute}
- Konfigurationsklasse

\paragraph{Methoden}

\begin{itemize}
\item + processTable(Table table, int step) : void
nimmt die Table entgegen und schreibt die Daten in Observations, step sagt wie viele zeilen auf einmal abgearbeitet werden, bevor observations hochgeladen werden
\item - createObs(String name, Object result, ZonedDateTime time) : Observation
\item - createObs(String name, Object result, List<String> time, List<Formatter> format, String zone) : Observation
methoden zum Erstellen von Observations mit verschiedenen Parametern
\item - combineResults(List<Object> results) : Object
erstellt aus mehreren results ein einziges result-Object
\end{itemize}