
\subsection{ObservationCreator}


Hier werden die Observations erstellt, die anschließend auf den Frost-Server hochgeladen werden können.
Dazu werden die Daten aus der internen Tabellendarstellung gelesen und daraus zeilenweise Observations erstellt, mit der Duplikatklasse auf Duplikate überprüft und anschließend hochgeladen

\paragraph{Attribute}
\paragraph{Methoden}

\begin{itemize}
\item \textit{public Observation create(Datastream stream, Object result, ZonedDateTime phenomenonTime)} Erstellt eine Observation aus Result und Beobachtungszeit für einen Datastream
\item \textit{public Observation create(MultiDatastream stream, Object result, ZonedDateTime phenomenonTime)} Erstellt eine Observation aus Result und Beobachtungszeit für einen MultiDatastream
\item \textit{public Observation create(Datastream stream, Object result, ZonedDateTime phenomenonTime, ZonedDateTime resultTime)} Erstellt eine Observation aus Result, Beobachtungszeit und Verarbeitungszeit für einen Datastream
\item \textit{public Observation create(MultiDatastream stream, Object result, ZonedDateTime phenomenonTime, ZonedDateTime resultTime)} Erstellt eine Observation aus Result, Beobachtungszeit und Verarbeitungszeit für einen MultiDatastream
\item \textit{public Object combineResults(List<Object> results)} Erstellt aus mehreren Objects ein einzelnes, um es als Result für eine Observation verwenden zu können
\end{itemize}