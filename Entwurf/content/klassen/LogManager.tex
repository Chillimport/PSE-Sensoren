\subsection{Logmanager}

\paragraph{Beschreibung}
Diese Klasse implementiert das Log-Fenster und hält es auf dem Laufenden.
Außerdem werden alle Imports in ein permanentes (seperates) Log geschrieben.


\paragraph{Attribute}

\begin{itemize}
	
\item \textit{private final static LogManager logManager}  
\\ Die einzige LogManager-Instanz 
\item \textit{private final static Logger log}
\\ Java-Logger Instanz
\item \textit{private static FileHandler txtFile}
\\ Text-File in die geschrieben wird
\item \textit{private static SimpleFormatter formatter}
\\ Text Formatter

\end{itemize}

\paragraph{Methoden}

\begin{itemize}
	
\item \textit{ private LogManager() : LogManager}  \\ Privater Konstruktor um mehrfache Erstellung zu unterbinden
\item \textit{ private initLog() : void} \\ Initialisiert einen Java-Logger
\item \textit{ public getLog() : Logger} \\ Gibt die Java-Logger Instanz zurück
\item \textit{ public getInstance() : LogManager}  \\ Verweist auf den einmaligen LogManager
\item \textit{ public writeToLog(String message, boolean severe) : void} \\ Schreibt <msg> als info in das Log wenn <severe> false ist,
sonst als severe.
	
\end{itemize}