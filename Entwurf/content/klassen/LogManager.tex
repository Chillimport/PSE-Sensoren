\subsection{Logmanager}

\paragraph{Beschreibung}
Diese Klasse implementiert das Log-Fenster und hält es auf dem Laufenden.
Außerdem werden alle Imports in ein permanentes (seperates) Log geschrieben.


\paragraph{Attribute}

\begin{itemize}
	
\item \textit{- final static LogManager logManager}  
\\ Die einzige LogManager-Instanz 
\item \textit{- Logger log}
\\ Java-Logger Instanz
\item \textit{- FileHandler txtFile}
\\ Text-File in die geschrieben wird
\item \textit{-  SimpleFormatter formatter}
\\ Text Formatter

\end{itemize}

\paragraph{Konstruktoren}
\begin{enumerate}[-]
	\item \textit{LogManager()} \\
	Privater Konstruktor um mehrfache Erstellung zu unterbinden
\end{enumerate}


\paragraph{Methoden}

\begin{enumerate}[+]

	\item \textit{static getInstance()}  \\ Verweist auf den einmaligen LogManager
	
	\vspace{-0.2cm}
	\begin{enumerate}[$\circ$]
		\item \textit{LogManager} \\ Die LogManager Instanz
	\end{enumerate}

\item \textit{writeToLog(String message, boolean severe)} \\ Schreibt <msg> als info in das Log wenn <severe> false ist,
sonst als severe.
\begin{enumerate}[$\bullet$]
\item \textit{string message} Die zu schreibende Nachricht
\item \textit{boolean severe} Ob die Nachricht dringend ist oder nicht
\end{enumerate}
\end{enumerate}


\begin{enumerate}[$-$]

\item \textit{initLog()} \\ Initialisiert einen Java-Logger
\vspace{-0.2cm}

\item \textit{getLogger()} \\ Gibt die Java-Logger Instanz zurück
\begin{enumerate}[$\circ$]
	\item Logger
\end{enumerate}
	