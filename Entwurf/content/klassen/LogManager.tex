\subsection{LogManager}

\paragraph{Beschreibung}
Diese Klasse implementiert das Log-Fenster und hält es auf dem Laufenden.
Außerdem werden alle Imports in ein permanentes (seperates) Log geschrieben.

\paragraph{Attribute}
\begin{enumerate}[$\bullet$]
	\item \underline{\textit{LOG_MANAGER: LogManager}}:  Die einzige LogManager-Instanz
	\item \textit{log: Logger} Die Java-Logger Instanz
	\item \textit{txtFile: FileHandler} Text-File in die geschrieben wird
	\item \textit{formatter: SimpleFormatter} Text-Formatter des Java-Logging-Pakets
\end{enumerate}

\paragraph{Konstruktoren}
\begin{enumerate}[-]
	\item \textit{LogManager()} \\
	Privater Konstruktor um mehrfache Erstellung zu unterbinden
\end{enumerate}


\paragraph{Methoden}

\begin{enumerate}[+]

	\item \underline{\textit{getInstance(): LogManager}} \\
	Verweist auf den einmaligen LogManager
	
	\vspace{-0.2cm}
	\begin{enumerate}[$\circ$]
		\item \textit{LogManager} \\
		Die LogManager-Instanz
	\end{enumerate}

	\item \textit{writeToLog(String message, boolean severe)} \\
	Schreibt <msg> als Info in das Log wenn <severe> false ist, sonst als severe.
	\begin{enumerate}[$\bullet$]
		\item \textit{String message} Die zu schreibende Nachricht
		\item \textit{boolean severe} Ob die Nachricht dringend ist oder nicht
	\end{enumerate}
\end{enumerate}


\begin{enumerate}[$-$]

	\item \textit{initLog(): void} \\ Initialisiert einen neuen Java-Logger
	\vspace{-0.2cm}

	\item \textit{getLogger() : Logger} \\
	Gibt die Java-Logger-Instanz zurück
	\begin{enumerate}[$\circ$]
		\item \textit{Logger}: Die Logger-Instanz
	\end{enumerate}
\end{enumerate}
	