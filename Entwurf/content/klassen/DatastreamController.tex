\subsection{DatastreamController}

\paragraph{Beschreibung}
Diese Controller-Klasse empfängt Anfragen vom Frontend, die das Erstellen oder Abrufen von (Multi-)Datastreams betreffen. Zur Bearbeitung dieser Anfragen greift die Klasse auf den Frost-Client zu.


\paragraph{Attribute}

\paragraph{Methoden}
\begin{itemize}
\item[+] \textit{createDatastream(String name, String description, ObservationType observationType, Sensor sensor, List<UnitOfMeasurement> units) : void }
erstellt mithilfe des Frost-Client einen neuen (Multi-)Datastream und lädt diesen auf den Frostserver.
\item[+] \textit{getDatastream(int id) : Thing}
sucht auf dem Server den (Multi-)Datastream mit der angegebenen id und gibt diesen zurück (NULL im Fehlerfall)
\item[+] \textit{getDatastreams(String begin) : List<Datastream> }
sucht auf dem Frost-Server die (Multi-)Datastreams (maximal die ersten 7), deren Namen mit der angegeben Zeichenkette beginnt.
\end{itemize}