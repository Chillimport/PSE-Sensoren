\subsection{<{<{interface}>}> Converter}
Konvertiert eine Datei aus dem File-Format in das Table-Format für die interne Repräsentation und wieder zurück \\

\paragraph{Methoden}
\begin{enumerate}[+]
	\item \underline{\textit{convert(File file, configuration cfg): Table}} \\
	Konvertiert eine Tabellen-Datei in eine Instanz der Klasse Table
	
	\begin{enumerate}[$\bullet$]
		\item \textit{File file}: Die zu konvertierende Datei, kann NULL sein, muss sonst aber eine Tabelle repräsentieren
		\item \textit{Configuration cfg}: Die zu nutzende Konfiguration
	\end{enumerate}
	\vspace{-0.2cm}
	\begin{enumerate}[$\circ$]
		\item \textit{Table}: die Table-Instanz die die Tabelle repräsentiert
	\end{enumerate}

	\item \underline{\textit{convertBack(Table table, Configuration cfg): File}} \\
	Konvertiert eine Table-Instanz wieder zurück in eine Datei, die gespeichert werden kann
	\begin{enumerate}[$\bullet$]
		\item \textit{Table table}: Die zu konvertierende Table-Instanz
		\item \textit{Configuration cfg}: Die zu nutzende Konfiguration
	\end{enumerate}
	\vspace{-0.2cm}
	\begin{enumerate}[$\circ$]
		\item \textit{File}: Die Datei
	\end{enumerate}
\end{enumerate}


\subsection{CSVConverter implements Converter}
Konvertiert nur CSV-Dateien. Implementierung der abstrakten Klasse \textit{Converter}. \\

\paragraph{Methoden}

\begin{enumerate}[+]
	\item \underline{\textit{convert(File file, Configuration cfg): Table}} \\
	Konvertiert eine Tabellen-Datei im CSV-Format in eine Instanz der Klasse Table
	
	\begin{enumerate}[$\bullet$]
		\item \textit{File file}: Die zu konvertierende Datei, kann NULL sein, muss ansonsten aber eine Tabelle im CSV-Format repräsentieren
		\item \textit{Configuration cfg}: Die zu nutzende Konfiguration, diese stellt u.a. die zu entfernenden Kopfzeilen, die Magic Numbers und das Trennsymbol bereit
	\end{enumerate}
	\vspace{-0.2cm}
	\begin{enumerate}[$\circ$]
		\item \textit{Table}: die Table-Instanz die die Tabelle repräsentiert
	\end{enumerate}
	
	\item \underline{\textit{convertBack(Table table, Configuration cfg): File}} \\
	Konvertiert eine Table-Instanz wieder zurück in eine Datei, die gespeichert werden kann
	\begin{enumerate}[$\bullet$]
		\item \textit{Table table}: Die zu konvertierende Table-Instanz
		\item \textit{Configuration cfg}: Die zu nutzende Konfiguration, die u.a. das Trennsymbol und die Kopfzeilen, die wieder hinzugefügt werden, bereitstellt
	\end{enumerate}
	\vspace{-0.2cm}
	\begin{enumerate}[$\circ$]
		\item \textit{File}: Die CSV-Datei
	\end{enumerate}

	\item \underline{\textit{removeHeaders(File file, Configuration cfg): File}} \\
	Entfernt die Kopfzeilen der CSV-Datei, welche durch die Konfiguration festgelegt sind
	\begin{enumerate}[$\bullet$]
		\item \textit{File file}: Die CSV-Datei von der die Kopfzeilen entfernt werden sollen
		\item \textit{Configuration cfg}: Die Konfiguration, die die Anzahl Kopfzeilen enthält
	\end{enumerate}
	\vspace{-0.2cm}
	\begin{enumerate}[$\circ$]
		\item \textit{File}: Die selbe CSV-Datei, nur ohne Kopfzeilen
	\end{enumerate}
\end{enumerate}

\subsection{ExcelConverter implements Converter}
Konvertiert nur Excel-Dateien. Implementierung der abstrakten Klasse \textit{Converter}. \\

\paragraph{Methoden}
\begin{enumerate}[+]
	\item \underline{\textit{convert(File file): Table}} \\
	Konvertiert eine Tabellen-Datei im Excel-Format in eine Instanz der Klasse Table.
	Jeder Wert der Excel-Tabelle behält seinen Datentyp bzw wird in den entsprechenden Java-Datentyp umgewandelt.
	Siehe auch: \href{https://support.office.com/en-us/article/data-types-in-data-models-e2388f62-6122-4e2b-bcad-053e3da9ba90#__toc327893213}{Datentypen in MS Excel} und 
	\href{http://poi.apache.org/spreadsheet/}{Apache POI Spreadsheet Library}	
	\begin{enumerate}[$\bullet$]
		\item \textit{File file}: Die zu konvertierende Datei, kann NULL sein, muss ansonsten aber eine Tabelle im Excel-Format repräsentieren
		\item \textit{Configuration cfg}: Die zu nutzende Konfiguration
	\end{enumerate}
	\vspace{-0.2cm}
	\begin{enumerate}[$\circ$]
		\item \textit{Table}: die Table-Instanz die die Tabelle repräsentiert
	\end{enumerate}
	
	\item \underline{\textit{convertBack(Table table, Configuration cfg): File}} \\
	Konvertiert eine Table-instanz wieder zurück in eine Excel-Datei, die gespeichert werden kann.
	Die Datentypen werden -wenn möglich- aus dem Java-Datentyp wieder in den Excel-Datentyp zurück-konvertiert.
	Siehe auch: \href{https://support.office.com/en-us/article/data-types-in-data-models-e2388f62-6122-4e2b-bcad-053e3da9ba90#__toc327893213}{Datentypen in MS Excel}
	\begin{enumerate}[$\bullet$]
		\item \textit{Table table}: Die zu konvertierende Table-Instanz
		\item \textit{Configuration cfg}: Die zu nutzende Konfiguration
	\end{enumerate}
	\vspace{-0.2cm}
	\begin{enumerate}[$\circ$]
		\item \textit{File}: Die Excel-Datei
	\end{enumerate}
\end{enumerate}
