\subsection{<{<abstract>}> Table} \label{Table}

\paragraph{Attribute}

\begin{itemize}
	\item[-] \textit{table: List{<List<Object>}>} \\
	Speichert die Daten der Tabelle, jedes Listenelement enthält eine weitere Liste, welche die Elemente der Tabelle hält. \\
	In der Implementierung der abstrakten Klasse muss entschieden werden welche Liste die Spalten und welche Liste die Zeilen enthält.\\
\end{itemize}

\paragraph{Methoden}

\begin{enumerate}[+]
	\item \textit{<{<abstract>}> getElement(Integer row, Integer column): Object} \\
	Gibt ein einzelnes Element der Tabelle an der angegebenen Stelle zurück
	\begin{enumerate}[$\bullet$]
		\item \textit{Integer row}: Die Zeile an der das Element liegt, darf nicht NULL sein
		\item \textit{Integer column}: Die Spalte an der das Element liegt, darf nicht NULL sein
	\end{enumerate}
	\vspace{-0.2cm}
	\begin{enumerate}[$\circ$]
		\item \textit{Object}: Das Element an der Stelle
	\end{enumerate}
	
	\item \textit{<{<abstract>}> setElement(Object o, Integer row, Integer column): void} \\
	Ersetzt ein Element der Tabelle mit einem neuen Element
	\begin{enumerate}[$\bullet$]
		\item \textit{Object o}: Das neue Element, darf NULL sein
		\item \textit{Integer row}: Die Zeile des Elements
		\item \textit{Integer column}: Die Spalte des Elements
	\end{enumerate}
	\vspace{-0.2cm}
	
	\item \textit{<{<abstract>}> remove(Integer row, Integer column): void} \\
	Ersetzt ein Element mit NULL
	\begin{enumerate}[$\bullet$]
		\item \textit{Integer row}: Die Zeile des Elements
		\item \textit{Integer column}: Die Spalte des Elements
	\end{enumerate}
	\vspace{-0.2cm}
	
	\item \textit{<{<abstract>}> getRowCount(): Integer} \\
	Gibt die Anzahl Zeilen zurück
	\vspace{-0.2cm}
	\begin{enumerate}[$\circ$]
		\item \textit{Integer}: Die Anzahl Zeilen
	\end{enumerate}

	\item \textit{<{<abstract>}> getColumnCount(): Integer} \\
	Gibt die Anzahl Spalten zurück
	\vspace{-0.2cm}
	\begin{enumerate}[$\circ$]
		\item \textit{Integer}: Die Anzahl Spalten
	\end{enumerate}

	\item \textit{<{<abstract>}> clone(): Table} \\
	Erstellt eine ganze Kopie der Tabelle
	\vspace{-0.2cm}
	\begin{enumerate}[$\circ$]
		\item \textit{Table}: die Kopie der Tabelle als neue Table-Instanz
	\end{enumerate}

	\item \textit{clear(): void} \\
	Löscht die Tabelle komplett und ersetzt sie gegen NULL

	\item \textit{isEmpty(): boolean} \\
	Gibt zurück ob die Tabelle leer ist
	\vspace{-0.2cm}
	\begin{enumerate}[$\circ$]
		\item \textit{boolean}: true wenn die Tabelle leer oder NULL ist
	\end{enumerate}

	\item \textit{<{<abstract>}> compare(Table o): boolean} \\
	Vergleicht die Table-Instanz mit einer anderen Instanz
	\begin{enumerate}[$\bullet$]
		\item \textit{Table o}: Die andere zu vergleichede Table-Instanz
	\end{enumerate}
	\vspace{-0.2cm}
	\begin{enumerate}[$\circ$]
		\item \textit{boolean}: true, wenn alle Einträge gleich sind, sonst false
	\end{enumerate}
\end{enumerate}