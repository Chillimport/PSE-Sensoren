\subsection*{CSVConverter implements Converter}\label{csvConv}
Konvertiert eine Datei aus dem File-Format in das Table-Format für die interne Repräsentation und wieder zurück \\
Konvertiert nur CSV-Dateien. \\

\paragraph{Methoden}

\begin{enumerate}[+]
	\item \underline{\textit{convert(File file, Configuration cfg): RowTable}} \\
	Konvertiert eine Tabellen-Datei im CSV-Format in eine Instanz der Klasse RowTable
	
	\begin{enumerate}[$\bullet$]
		\item \textit{File file}: Die zu konvertierende Datei, kann NULL sein, muss ansonsten aber eine Tabelle im CSV-Format repräsentieren
		\item \textit{Configuration cfg}: Die zu nutzende Konfiguration, diese stellt u.a. die zu entfernenden Kopfzeilen, die Magic Numbers und das Trennsymbol bereit
	\end{enumerate}
	\vspace{-0.2cm}
	\begin{enumerate}[$\circ$]
		\item \textit{RowTable}: die RowTable-Instanz die die Tabelle repräsentiert
	\end{enumerate}
	
	\item \underline{\textit{convertBack(RowTable table, Configuration cfg): File}} \\
	Konvertiert eine RowTable-Instanz wieder zurück in eine Datei, die gespeichert werden kann
	\begin{enumerate}[$\bullet$]
		\item \textit{RowTable table}: Die zu konvertierende RowTable-Instanz
		\item \textit{Configuration cfg}: Die zu nutzende Konfiguration, die u.a. das Trennsymbol und die Kopfzeilen, die wieder hinzugefügt werden, bereitstellt
	\end{enumerate}
	\vspace{-0.2cm}
	\begin{enumerate}[$\circ$]
		\item \textit{File}: Die CSV-Datei
	\end{enumerate}

	\item \underline{\textit{removeHeaders(File file, Configuration cfg): File}} \\
	Entfernt die Kopfzeilen der CSV-Datei, welche durch die Konfiguration festgelegt sind
	\begin{enumerate}[$\bullet$]
		\item \textit{File file}: Die CSV-Datei von der die Kopfzeilen entfernt werden sollen
		\item \textit{Configuration cfg}: Die Konfiguration, die die Anzahl Kopfzeilen enthält
	\end{enumerate}
	\vspace{-0.2cm}
	\begin{enumerate}[$\circ$]
		\item \textit{File}: Die selbe CSV-Datei, nur ohne Kopfzeilen
	\end{enumerate}
\end{enumerate}
