\documentclass[12 pt]{article}

\usepackage[ngerman]{babel}
\usepackage[utf8]{inputenc}
\begin{document}


\subsection{Produktübersicht}
Was ist der Frost-Server?
Weboberfläche zum einfachen Abspeichern von Sensordaten im FROST Server

\subsection{Musskriterien}
\begin{enumerate}
\item Import von csv/xlsx dateien
\item konvertierung von sensordaten von csv und xlsx in ISO format nach Vorgabe der SensorThings API (Format der Daten auf Sever)
\item Konfiguration der Formate der Sensordaten
\item erstellen, abspeichern, auslesen von Favoriten-konfigurationen
\item Datentyptransformationen
\item Verarbeiten von Sonderwerten (Magic Numbers)
\item Import von Fremdquellen wie anderen webseiten
\item Weitergabe von Fehlermeldungen an Nutzer, falls Import schief läuft
\end{enumerate}

\subsection{Wunschkriterien}
\begin{enumerate}
\item Bereitstellen einer vorauswahl (default)
\item überprüfung auf Duplikate
\item Erkennen und Rückgabe von Tipp- und Sinnfehlern bzw. Vorschläge für weiteres Vorgehen
\item Rückgabe nicht bearbeiteter Datensätze in neuer CSV-Datei
\item Docker-Container des Server-Programmes
\item Auswahl komplexer Transformationen im Webinterface (Aggregationen (Zusammenlegen von Daten), Summe, Min/Max, Durchschnitt...)
\item automatisierter regelmäßiger Download von entfernter Adresse
\item Import kompletter Datensätze von anderem FROST-Server
\item automatisierte Erkennung des Formats
\item Erweiterungsmöglichkeit der Software für besondere Formate

\end{enumerate}

\subsection{Abgrenzungskriterien}
keine verwaltung bestehender Sensordaten auf dem Server. Produkt dient nur zum Import von Sensordaten

\subsection{Anwendungsbereiche}
Überll wo Daten im CSV-Format verarbeitet werden müssen: \\
\begin{itemize}
\item Verarbeiten von neuen Sensordaten im CSV-Format
\item Integration von bestehenden Sensordaten in den FROST-Server
\end{itemize}

\subsection{Zielgruppen}
2 Zeilgruppen:
\begin{itemize}
\item Besitzer von Sensoren; Privatpersonen
\item Personen ohne technisches Vorwissen bspw. Beamte
\item Personen mit technischem know-how
\end{itemize}

\subsection{Betriebsbedingungen}
Unterschieden wird zwischen Endgerät und Server. \\
Die Betriebsbedingungen beschreiben die minimalen Anforderungen die die Software benötigt.
\begin{itemize}
	\item Browser
	\item JavaScript
	\item Verbindung zum FROST-Server und zum Programmserver
\end{itemize}
\begin{itemize}
	\item Verbindung zum FROST-Server
	\item Java JRE
	\item Anbindung an Endgerät (zB Internet)
\end{itemize}

\subsection{Produktumgebung}
Die Produktumgebung ist eine Instanz unter der die Software läuft. Diese erfüllt mindestens die Betriebsbedingungen.
\begin{itemize}
\item head-less Server
\item Linux/Windows Server
\item Java JRE
\item SensorThings API
\end{itemize}

\subsection{Funktionale Anforderungen}
\begin{itemize}
\item Datei laden: Durch Klick auf den "Datei auswählen"-Button öffnet sich der Datei-auswählen-Systemdialog, bei dem man die Datei auswählen kann. Im Systemdialog wird das Dateiformat auf entweder CSV oder XLS/XSLX begrenzt. Nachdem die Datei im Systemdialog ausgewählt wurde wird sie auf die Größe überprüft und bei erfolgreicher Prüfung  hochgeladen.
\item Domain angeben: In das Textfeld wird die Domain des FROST-Servers eingegeben. Die Domain wird überprüft ( Verbindung? Zugriffsrechte? Ist es ein FROST-Server?) und durch einen Haken bestätigt
\item Konfiguration laden: Durch Klick auf "Config auswählen" öffnet sich der Datei-auswählen-Systemdialog in dem man die CFG-Konfigurationsdatei auswählen kann. Die Konfiguration wird geprüft und anschließend geladen.
\item Konfiguration speichern: Durch Klick auf den Button wird der Download der aktuellen Konfiguration als CFG-Datei angestoßen
\item Zurücksetzen: Dabei wird die derzeitige Konfiguration zurückgesetzt

\item ?-Button / Help-Button: Durch Klick wird die README-Datei im Browser aufgerufen
\item 
\end{itemize}


\subsection{Produktdaten}
\begin{itemize}
\item Abspeichern der Konfigurationen
\item Speichern der Domain des FROST-Servers auf dem Server
\item Speichern der Domain der entfernten CSV-Adresse auf dem Server
\end{itemize}

\subsection{Produktleistungen}
\begin{itemize}
\item Statusmeldungen an den Nutzer
\item Schnelle Reaktionszeit auf Benutzer-Eingaben
\item korrekte Übertragung der Daten
\item Robustheit gegenüber falschen Formatangeben (Fehlermeldungen)
\end{itemize}

\subsection{Nichtfunktionale Anforderungen}
einzuhaltende Gesetze, Normen, Urheber- und Markenrechte,
Sicherheitsanforderungen, Plattformabhängigkeiten
\begin{itemize}
\item Genutzt wird der FROST-Server und die SensorThings API (beides Open-Source)
\item Die Software ist Open-Source (GPLv3)
\item Die Software ist abhängig von der Java-Plattform
\item Robustheit gegenüber zu großen bzw. zu vielen Anfragen
\item HTTPS-/SSL-Verschlüsselung
\end{itemize}


\subsection{Qualitätsanforderungen}
In diesem Kapitel wird den Qualitätsmerkmalen Funktionalität, Zuverlässigkeit, Benutzbarkeit, Effizienz, Änderbarkeit und Übertragbarkeit je eine Qualitätsstufe aus sehr gut, gut, normal und nicht relevant zugeordnet.
\begin{itemize}
\item Die Software sollte zuverlässig auch über längere Zeiträume ohne Unterbrechung laufen sofern es die Produktumgebung zulässt
\item Die Benutzbarkeit hat höchste Priorität, es soll auch Nutzern mit wenig Computerkentnissen möglich sein die Software zu bedienen
\item Es soll einfach möglich sein, Änderungen und Erweiterungen an der Software vorzunehmen
\item Die Software ist einfach zu übertragen, da sie auf der plattformunabhängigen Programmiersprache Java basiert, es wird nur eine Produktumgebung vorausgesetzt, die lediglich die Betreibsbedingungen erfüllt
\end{itemize}


\subsection{Globale Testfälle und Szenarien / Anwendungsfälle}
\begin{itemize}
\item to be added after the GUI design
\end{itemize}


\subsection{Zeitplanung}
Beispielhafte Terminplanung: \\
\begin{itemize}
	\item 07.05. - 27.05.: Pflichtenheft
	\item 28.05. - 24.06.: Entwurfsphase
	\item 25.06. - 22.07.: Implementierung
	\item 23.07. - 12.08.: Beispiel-Urlaubstermin (2 Wochen)
	\item 13.08. - 02.09.: Qualitätssicherung
	\item 03.09. - 09.09.: Terminfenster interne Abnahme
	\item 10.09. - 17.09.: Terminfenster Abschlusspräsentation
\end{itemize}




Fragen:

\begin{itemize}
\item Anfragen an server für POST, PATCH, PUT,...
\item Standard der Daten (wir abgespeichert?), Beispiel CSV Dateien,...
\item welche Formate sind möglich?
\item was sind komplexere Tranformationen?
\item was passiert mit falschen Daten? (13. Monat)
\end{itemize}


Glossar:
\begin{itemize}
\item FROST
\item SensorThings
\end{itemize}


\end{document}