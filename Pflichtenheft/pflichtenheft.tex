\documentclass[12 pt]{article}

\usepackage[ngerman]{babel}
\usepackage[utf8]{inputenc}

\begin{document}

%TODO
\subsection{Produktübersicht}
Unsere Software stellt ein Webinterface zum einfachen Import von Sensordaten aus Tabellen auf einen FROST-Server bereit. \\
Der FROST-Server vom Fraunhofer IOSB ist eine Open-Source-Implementierung des SensorThings-Standards des OGC und unter %TODO Github Link
downloadbar. Unsere Software stellt zu diesem 


\subsection{Musskriterien}
\begin{enumerate}
\item Import von Tabellen \\
	Die Daten sollen aus Tabellen im CSV-/XSLX-Format importiert werden
\item Konvertierung des Formats der Sensordaten \\
	Die Daten sollen aus der Tabelle in eine einheitliches Format nach Vorgabe der SensorThings API konvertiert werden. Damit entsprechen sie dem Format der Daten auf einem FROST-Server.
\item Erstellen, abspeichern, laden von Konfigurationen \\
	Die Konfiguration der Software (ausgew. Thing, Server, Format) soll gespeichert und dem Nutzer via Download einer CFG-Datei bereitgestellt werden. Eine Konfiguration soll später auch wieder geladen werden können.
\item Datentyptransformationen \\
	Ein Wert als bspw. String soll nicht als String auf den FROST-Server übertragen werden sollen, sondern von der Software vorher in einen passenden Datentyp (bswp. Integer) umgewandelt werden
\item Verarbeiten von Sonderwerten (Magic Numbers) \\
	Diese Werte sollen von der Software ignoriert und gegen einen NULL-Wert (o.ä.) ausgetauscht werden, bevor sie auf den FROST-Server übertragen werden.
\item Import von Fremdquellen wie anderen Webseiten \\
	Es soll möglich sein als Quelldatei auch einen entfernten Speicherort -zum Beispiel einen Weblink- anzugeben.
\item Weitergabe von Fehlermeldungen an Nutzer, falls Import schief läuft \\
	Bei Fehlern in der Verarbeitung oder in der Übertragung soll der Nutzer eine Rückmeldung kriegen
\item Logging \\
	Sämtliche Eingaben und Interaktionen des Nutzers mit der Software sollen in einer Log-Datei gespeichert werden
\end{enumerate}

\subsection{Wunschkriterien}
\begin{enumerate}
\item Sprachauswahl: Deutsch/Englisch
\item Bereitstellen einer Vorauswahl (default-Konfiguration) \\
	Dem Benutzer wird eine vorgefertigte Standard-Konfiguration bereitgestellt.
\item Überprüfung auf Duplikate \\
	Beim Upload der Daten auf den FROST-Server werden diese mit den bestehenden Daten abgeglichen und Duplikate werden nicht erneut hochgeladen.
\item Korrektur von falschen Daten \\
	Der Benutzer hat dei Möglichkeit Fehler in den Daten direkt korrigieren.
\item Rückgabe nicht bearbeiteter Datensätze in neuer CSV-Datei \\
	Bei (teilweise) fehlgeschlagener Übertragung zum FROST-Server werden die nicht übertragenen Daten in einer CSV-Datei an der Benutzer zurückgegeben.
\item Docker-Container \\
	Zur einfachen Installation steht das Programm als Docker-Container zur Verfügung
\item Auswahl komplexer Transformationen Aggregationen (Zusammenlegen von Daten)
	Aggregationen wie Summe, Minimum/Maximum, Durchschnitt, ... über ausgewählte Daten werden dem Benutzer in einem Webinterface bereitgestellt.
\item automatisierte Daten-Aktualisierung \\
	Bei Import von Daten von einer entfernten Adresse wird eine automatische regelmäßige Übertragung bereitgestellt.
\item Import kompletter Datensätze von anderem FROST-Server \\
	Der Benutzer hat zusätzlich die Möglichkeit, Daten von einem anderen FROST-Server 
\item automatisierte Erkennung des Formats \\
	Aus der CSV-/XLSX-Datei werden Formate automatisch erkannt.


\end{enumerate}

\subsection{Abgrenzungskriterien}
\begin{enumerate}
\item Keine Verwaltung von Sensor-Daten auf dem FROST-Server \\
	Die Software soll keine schon auf dem FROST-Server liegenden Daten verwalten, ändern oder löschen
\end{enumerate}


\subsection{Anwendungsbereiche}
Überall wo Daten im CSV-Format verarbeitet werden müssen: \\
Sie Software soll überall dort eingesetzt werden wo Daten aus Tabellen importiert, konvertiert und auf den FROST-Server übertragen werden sollen. \\
Hierzu zählt:
\begin{itemize}
\item Der Import von neuen Sensordaten, welche in CSV-/XSLX-Tabellen-Form vorliegen, in den FROST-Server
\item Die Integration von bestehenden (archivierten) Sensordaten, welche in CSV-/XSLX-Tabellen-Form vorliegen, in den FROST-Server
\end{itemize}

%TODO Die zwei Zielgruppen genauer darstellen/abgrenzen
\subsection{Zielgruppen}
Die Software ist auf zwei Zielgruppen ausgelegt.\\
Zum einen gibt es die Personen, welche einfach Sensordaten auf einen Frostserver laden möchten,  ohne die Konfigurationen zu ändern. Der Fokus liegt bei dieser Gruppe auf der einfachen Bedienung der Software.\\
Zum anderen gibt es die Nutzergruppe welche ihre eigenen Konfigurationen erstellen. Für diese Nutzergruppe ist es wichtig, dass sie komplexe Einstellungsmöglichkeiten in den Konfigurationen nutzen können.

\subsection{Betriebsbedingungen}
Unterschieden wird zwischen Endgerät und Server. \\
Die Betriebsbedingungen beschreiben die minimalen Anforderungen die die Software zur Ausführung benötigt. \\
Das Ziel-Endgerät ist ein Computer/Tablet, wobei jedes Gerät mit einer Internetanbindung zum Programmserver, einem Webbrowser und aktiviertem JavaScript ausreichend ist.\\
\ \\
Der Programmserver benötigt nur eine Anbindung an den FROST-Server und das Endgerät und ein Java JRE.
\begin{itemize}
	\item Browser
	\item JavaScript
	\item Verbindung zum Programmserver
\end{itemize}
\begin{itemize}
	\item Verbindung zum FROST-Server
	\item Java JRE
	\item Anbindung an Endgerät (zB Internet)
\end{itemize}

\subsection{Produktumgebung}
Die Produktumgebung ist eine Instanz unter der die Software läuft. Diese erfüllt mindestens die Betriebsbedingungen.
\begin{itemize}
\item head-less Server
\item Linux/Windows Server
\item Java JRE
\item SensorThings API
\end{itemize}
Die Software soll plattformunabhängig auf einem Server mit minimaler Rechen- und Speicherkapazität laufen. Des weiteren soll das Programm head-less ununterbrochen laufen. Der Server stellt das Java JRE zur Verfügung, auf dem die Software läuft. Die Software benötigt nur noch eine Verbindung zu einer FROST-Server-Instanz, welche ihm vom Nutzer in der Konfiguration gegeben wird.

\subsection{Funktionale Anforderungen}

%Vorschlag:

\begin{itemize}
\item Kurzanleitung auf der Weboberfläche: Auf der Weboberfläche existiert die Möglichkeit, eine Anleitung angezeigt zu bekommen. Diese soll kurz beschreiben, wie der Konfigurator und die Weboberfläche an sich funktionieren.
\item Auswahlmöglichkeit des Zielservers: Auf der Weboberfläche soll ein ereichbarer Frost-Server zum Hinterlegen der Daten einstellbar sein
\item Auswahlmöglichkeit der Datenquelle: Auf der Weboberfläche lässt sich auswählen, ob eine Datei (.csv/.xlsx) hochgeladen, eine entfernte Web-Adresse als Quelle dient oder Daten aus einem schon bestehenden Server importiert werden sollen.
\item Auswahlmöglichkeit einer Konfiguration: Auf der Weboberfläche besteht die Möglichkeit eine Konfigation abzuspeichern und eine abgespeicherte Konfiguration erneut zu laden.
\item Bereitstellen einer Default-Konfiguration
\item Konfiguration zum Laden der Daten auf den Frost-Server: Auf der Weboberfläche lassen sich das Format der hochzuladenden Daten einstellen. Dazu wird die Auswahl bzw. das Neuanlegen eines Things und zugehörigen Datastreams ermöglicht. Außerdem lässt sich das Datumsformat in der Daten in der Quelldatei festlegen.
\item Hochladen der Daten auf den Frost-Server
\item Formatieren der Daten: Vor dem Abspeichern auf dem Frost-Server werden die Daten formatiert, sodass sie dem OGC SensorThings API Standard enstprechen.
\item Anlegen einer Log-Datei: Während der Nutzung der Weboberfläche wird eine Log-Datei erstellt, die sämtliche Aktionen des Nutzers sowie Weboberfäche/Server-Interaktionen aufzeichnet, um Fehlererkennung und -verfolgung zu erleichtern.
\item Sprachauswahl
\item Auswahlmöglichkeit komplexer Operationen auf Daten: Auf der Weboberfläche lassen sich komplexere Datentransformationen auswählen. Diese sollen das Abspeichern von Datenaggregationen wie Summe, MIn, Max und Durchschnitt des Datensatzes ermöglichen
\item Duplikaterkennung: Bevor Daten auf den Frost-Server hochgeladen werden, wird überprüft, ob diese schon existieren, um redundante Daten zu vermeiden.
\item Fehlermeldungen: Kommt es bei der Verarbeitung der Daten zu Problemen, ermöglichen aussagekräftige Fehlermeldungen eine effiziente Korrektur.
\item Rückgabe nicht verarbeiteter Daten: Können Teile des Datensatzes nicht verarbeitet werden, wird eine csv-Datei mit allen fehlerhaften Daten erstellt und an den Nutzer zurückgegeben.
\end{itemize}

\newpage

\begin{itemize}
%TODO GUI-Design von Thing-/Datastream-Erstellung
	%TODO Units Of Measurement bei Erstellung von Datastreams
%TODO Funk. Anf. Thing-/Datastream-Erstellung


\item Hilfe-Button: Durch Klick wird die README-Datei im Browser aufgerufen
\item Speichern-Button: Bei Klick auf diesen Button wird der Upload der Daten in die Datastreams/MultiDatastreams auf den FROST-Server angestoßen

\item Radio-Boxen "Datei auswählen"/"Web-Adresse angeben": Durch Auswahl einer der Boxen wird entweder die Textbox zur Eingabe des Weblinks der entfernten Datei freigeschaltet oder der "Durchsuchen"-Button zur Auswahl einer lokalen
\item Durchsuchen: Durch Klick auf den "Durchsuchen"-Button öffnet sich der Datei-auswählen-Systemdialog, bei dem man die Datei auswählen kann. Im Systemdialog wird das Dateiformat auf entweder CSV oder XLS/XSLX begrenzt. Nachdem die Datei im Systemdialog ausgewählt wurde, wird sie auf die Größe überprüft und bei erfolgreicher Prüfung hochgeladen.

\item Textbox "Ziel": Hier wird die Adresse des FROST-Servers angegeben
\item Button "Auswählen": Durch Klick auf diesen Button wird die im Textfeld "Ziel" angegebene Adresse überprüft und durch einen Haken und im Log bestätigt. (Es sollte überprüft werden ob eine Verbindung hergestellt werden kann,ob eventuell Zugriffsrechte benötigt werden und ob die Zieladresse überhaupt zu einem FROST-Server führt) 

\item Konfiguration durchsuchen/laden: Durch Klick auf "Durchsuchen" öffnet sich der Datei-auswählen-Systemdialog in dem man die CFG-Konfigurationsdatei auswählen kann. Die Konfiguration wird geprüft und anschließend geladen.
\item Konfiguration speichern: Durch Klick auf den Button "Speichern" wird der Download der aktuellen Konfiguration als CFG-Datei gestartet
\item Reset-Button: Bei Klick auf den Reset-Button wird die gesamte Konfiguration (Datums-/Zeitformat, Datastreams, Ziel) zurückgesetzt

\item Thing- / id-Dropdown-Menü: Ein existierendes Thing kann anhand des Namens oder der ID aus einer Liste ausgewählt werden
\item Button "Neu": Existiert ein Thing noch nicht so kann es durch den Button "neu" erstellt werden. Es öffnet sich eine Dialogbox, die es ermöglicht ein neues Thing zu erstellen

\item Textfelder "Spalte": Hier kann durch Angabe der Spaltennummern eingestellt werden in welchen Spalten das Datum/die Uhrzeit zu finden ist
\item Textfelder "Format": Hier wird per RegEx-String eingestellt, in welchem Format das Datum in den Spalten vorliegt
\item Button "+": Hier können mehr Spalten, die das Datum beinhalten, ausgewählt werden

\item "Name"-Dropdown-Menü: Hier kann der (Multi-)Datastream über seinen Namen oder seiner ID aus einer Liste ausgewählt werden
\item Spalten-Textfelder: Hier kann zu der vorangehenden Einheit die Spalte, die den Wert enthält, angegeben werden
\item Button "Neu": Durch Klick auf den Button öffnet sich ein Dialogfenster, welches das Erstellen eines neuen (Multi-)Datastreams ermöglicht
\item Button "+": Durch klick auf den Button "+" wird eine neue Box von Feldern erstellt, in der man Spalten auf Werte eines Datastreams abbilden kann
\end{itemize}

%TODO Nummerierung der Wunschkriterien in Fließtext
\subsection{Produktdaten}
Die gesicherten Daten der Software sind minimal, konkret heißt das es wird nur ein Log-File gesichert. Die Konfiguration wird dem Nutzer per CFG-Datei zum Download bereitgestellt und nicht Serverseitig gespeichert.\\
Sollte das Kriterium [Wunsch: Automatischer Download] mit aufgenommen werden, so muss die Adresse der entfernten Datei und des zu nutzenden FROST-Servers auf dem Server gespeichert werden.
\begin{itemize}
\item Log-Datei
\item Abspeichern der Konfigurationen
\item Speichern der Domain des FROST-Servers auf dem Server
\item Speichern der Domain der entfernten CSV-Adresse auf dem Server
\end{itemize}

\subsection{Produktleistungen}
Die Software soll sämtliche Aktionen der Nutzer in einem Log-File speichern. Die Eingaben sollen möglichst direkt verarbeitet und die Reaktionszeit möglichst gering sein. Bei langsamen Operationen soll der Nutzer Feedback über eine Fortschrittsleiste und über das Log-Fenster bekommen. \\
Die Daten müssen korrekt an den FROST-Server übermittelt werden, dazu gehört auch Robustheit der Anwendung gegenüber falschen Angaben. Bei falschen Angaben oder sonstigen Fehlern soll der Nutzer mittels einer Fehlermeldung benachrichtigt werden.
\begin{itemize}
\item Logging
\item Statusmeldungen an den Nutzer
\item Schnelle Reaktionszeit auf Benutzer-Eingaben
\item korrekte Übertragung der Daten
\item Robustheit gegenüber falschen Formatangeben (Fehlermeldungen)
\end{itemize}

\subsection{Nichtfunktionale Anforderungen}
Die Software ist Open-Source und benötigt einen erreichbaren FROST-Server. Sie basiert auf der SensorThings-API, welche wie der FROST-Server ebenfalls Open-Source ist.
Das Programm wird in Java geschrieben und hängt dementsprechend von einer Java-Umgebung ab.
Um Sicherheit zu gewährleisten sollte das Programm robust gegenüber schädlichen Anfragen sein, vor allem zu große oder zu viele Anfragen sollten dementsprechend abgelehnt werden um eine Überlastung des Servers zu vermeiden.
Außerdem sollten die daten HTTPS-/SSL-Verschlüsselt werden um Abfangen der daten zu verhindern.

(einzuhaltende Gesetze, Normen, Urheber- und Markenrechte,
Sicherheitsanforderungen, Plattformabhängigkeiten)
\begin{itemize}
\item Genutzt wird der FROST-Server und die SensorThings API (beides Open-Source)
\item Die Software ist Open-Source (GPLv3)
\item Die Software ist abhängig von der Java-Plattform
\item Robustheit gegenüber zu großen bzw. zu vielen Anfragen 		
\item HTTPS-/SSL-Verschlüsselung								
\end{itemize}


\subsection{Qualitätsanforderungen}
Das Programm sollte über lange Zeiträume auch ohne Unterbrechungen zuverlässig ausgeführt werden können, solange die Produktumgebung dies zulässt.
Die Benutzbarkeit hat höchste Priorität, auch Nutzer ohne oder wenig technischem Vorwissen sollten in der Lage sein die Software zu bedienen.
Außerdem soll es möglichst einfach sein die Software schnell und unkompliziert zu ändern oder zu erweitern.
Eine einfache Portierung ist gegeben, da die Software auf Java basiert und damit Platformunabhängig ist. Es wird lediglich eine Produktumgebung vorausgesetzt die die Betriebsbedingungen erfüllt.

(In diesem Kapitel wird den Qualitätsmerkmalen Funktionalität, Zuverlässigkeit, Benutzbarkeit, Effizienz, Änderbarkeit und Übertragbarkeit je eine Qualitätsstufe aus sehr gut, gut, normal und nicht relevant zugeordnet.)
\begin{itemize}
\item Die Software sollte zuverlässig auch über längere Zeiträume ohne Unterbrechung laufen sofern es die Produktumgebung zulässt
\item Die Benutzbarkeit hat höchste Priorität, es soll auch Nutzern mit wenig Computerkentnissen möglich sein die Software zu bedienen
\item Es soll einfach möglich sein, Änderungen und Erweiterungen an der Software vorzunehmen
\item Die Software ist einfach zu übertragen, da sie auf der plattformunabhängigen Programmiersprache Java basiert, es wird nur eine Produktumgebung vorausgesetzt, die lediglich die Betreibsbedingungen erfüllt
\end{itemize}

\subsection{Anwendungsszenarien}
Ideen für ANwendungsszenarien:
\begin{itemize}
\item Nutzer ohne technische Kenntnisse des Frost Server Aufbaus besitzt einen Frost Server und einen Sensor, der schon auf dem Frost Server abgespeichert ist. Außerdem hat er eine csv-Datei mit Daten vom Sensor, der einen schon bestehenden Datensatz des Sensors auf dem Server erweitern soll. Er navigiert in seinem Browser zur Web-oberfläche und stellt seine Datei als Quelle und den Frost Server als Ziel ein. Dann wählt er eine Favoritenkonfiguration aus, die zu seinem Sensor passt und lädt seine Daten durch Klick des "Upload"-Buttons hoch. Die Daten werden formatiert und auf dem Server abgrspeichert.
\item das gleiche mit einem erfahrenen Nutzer, der ein neues Thing, einen Sensor und zugehörigen Datastream erstellt
\item Szenario zum Abspeichern einer Konfiguration
\end{itemize}


\subsection{Globale Testfälle und Szenarien / Anwendungsfälle}
\begin{itemize}
%TODO to be added after the GUI design
\item to be added after the GUI design
\end{itemize}


\subsection{Zeitplanung}
Beispielhafte Terminplanung: \\
\begin{itemize}
	\item 07.05. - 27.05.: Pflichtenheft
	\item 28.05. - 24.06.: Entwurfsphase
	\item 25.06. - 22.07.: Implementierung
	\item 23.07. - 12.08.: Beispiel-Urlaubstermin (2 Wochen)
	\item 13.08. - 02.09.: Qualitätssicherung
	\item 03.09. - 09.09.: Terminfenster interne Abnahme
	\item 10.09. - 17.09.: Terminfenster Abschlusspräsentation
\end{itemize}


Glossar:
\begin{itemize}
\item FROST
\item SensorThings
\item CSV-/XSLX-Dateien
\item CFG-Datei
\item Thing
\item Integer
\item NULL
\item Magic Numbers
\item Log-Datei
\item Docker
\item JavaScript
\item Java JRE
\item head-less Server
\item OGC
\item README
\item Datastram
\item Regex
\item HTTPS/SSL
\item GPLv3
\item GUI 
\end{itemize}


\end{document}